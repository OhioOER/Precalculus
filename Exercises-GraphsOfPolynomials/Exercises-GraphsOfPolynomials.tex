\documentclass{ximera}
%% You can put user macros here
%% However, you cannot make new environments

%\listfiles

%\graphicspath{{./}{firstExample/}{secondExample/}}
\graphicspath{{./}{graphicsAndVideos/}}


\usepackage{tikz}
\usepackage{tkz-euclide}
\usepackage{tikz-3dplot}
\usepackage{tikz-cd}
\usetikzlibrary{shapes.geometric}
\usetikzlibrary{arrows}
\usetikzlibrary{decorations.pathmorphing,patterns}
\usetikzlibrary{backgrounds} % added by Felipe
\usetkzobj{all}
\pgfplotsset{compat=1.13} % prevents compile error.

\renewcommand{\vec}[1]{\mathbf{#1}}
\newcommand{\RR}{\mathbb{R}}
\newcommand{\dfn}{\textit}
\newcommand{\dotp}{\cdot}
\newcommand{\id}{\text{id}}
\newcommand\norm[1]{\left\lVert#1\right\rVert}
\newcommand{\dst}{\displaystyle}
 
\newtheorem{general}{Generalization}
\newtheorem{initprob}{Exploration Problem}

\tikzstyle geometryDiagrams=[ultra thick,color=blue!50!black]

\usepackage{mathtools}

% Added by precalculus team.
\usepackage{longtable}
\usepackage{multicol}

\newcounter{HW}

\author{The Authors}

\title{Graphs of Polynomials}

\begin{document}
\begin{abstract}
Exercises about graphs of polynomials
\end{abstract}
\maketitle

\renewcommand{\theenumi}{\arabic{enumi}.}

In Exercises \ref{polyfactsfirst} - \ref{polyfactslast}, find the degree, the leading term, the leading coefficient, the constant term and the end behavior of the given polynomial.

%\begin{multicols}{2}
\begin{enumerate}

\item  $f(x) = 4-x-3x^2$ \label{polyfactsfirst}
\item  $g(x) = 3x^5 - 2x^2 + x + 1$

\setcounter{HW}{\value{enumi}}
\end{enumerate}
%\end{multicols}

%\begin{multicols}{2}
\begin{enumerate}
\setcounter{enumi}{\value{HW}}

\item $q(r) = 1 - 16r^{4}$
\item $Z(b) = 42b - b^{3}$

\setcounter{HW}{\value{enumi}}
\end{enumerate}
%\end{multicols}


%\begin{multicols}{2}
\begin{enumerate}
\setcounter{enumi}{\value{HW}}

\item $f(x) = \sqrt{3}x^{17} + 22.5x^{10} - \pi x^{7} + \frac{1}{3}$
\item $s(t) = -4.9t^{2} + v_{\mbox{\tiny $0$}}t + s_{\mbox{\tiny $0$}}$

\setcounter{HW}{\value{enumi}}
\end{enumerate}
%\end{multicols}

%\begin{multicols}{2}
\begin{enumerate}
\setcounter{enumi}{\value{HW}}

\item $P(x) = (x - 1)(x - 2)(x - 3)(x - 4)$
\item $p(t) = -t^2(3 - 5t)(t^{2} + t + 4)$

\setcounter{HW}{\value{enumi}}
\end{enumerate}
%\end{multicols}

%\begin{multicols}{2}
\begin{enumerate}
\setcounter{enumi}{\value{HW}}

\item $f(x) = -2x^3(x+1)(x+2)^2$
\item $G(t) = 4(t-2)^2\left(t+\frac{1}{2}\right)$ \label{polyfactslast}

\setcounter{HW}{\value{enumi}}
\end{enumerate}
%\end{multicols}

%\end{document}

\phantomsection
\label{polygraphexercise}

In Exercises \ref{zeromultgraphfirst} - \ref{zeromultgraphlast}, find the real zeros of the given polynomial and their corresponding multiplicities.  Use this information along with a sign chart to provide a rough sketch of the graph of the polynomial.  Compare your answer with the result from a graphing utility.

%\begin{multicols}{2}
\begin{enumerate}
\setcounter{enumi}{\value{HW}}

\item $a(x) = x(x + 2)^{2}$ \label{zeromultgraphfirst}
\item $g(x) = x(x + 2)^{3}$

\setcounter{HW}{\value{enumi}}
\end{enumerate}
%\end{multicols}


%\begin{multicols}{2}
\begin{enumerate}
\setcounter{enumi}{\value{HW}}

\item $f(x) = -2(x-2)^2(x+1)$
\item $g(x) = (2x+1)^2(x-3)$

\setcounter{HW}{\value{enumi}}
\end{enumerate}
%\end{multicols}


%\begin{multicols}{2}
\begin{enumerate}
\setcounter{enumi}{\value{HW}}

\item $F(x) = x^{3}(x + 2)^{2}$
\item $P(x) = (x - 1)(x - 2)(x - 3)(x - 4)$

\setcounter{HW}{\value{enumi}}
\end{enumerate}
%\end{multicols}


%\begin{multicols}{2}
\begin{enumerate}
\setcounter{enumi}{\value{HW}}

\item $Q(x) = (x + 5)^{2}(x - 3)^{4}$
\item $h(x) = x^2(x-2)^2(x+2)^2$

\setcounter{HW}{\value{enumi}}
\end{enumerate}
%\end{multicols}


%\begin{multicols}{2}
\begin{enumerate}
\setcounter{enumi}{\value{HW}}

\item $H(t) = (3-t)(t^2+1)$
\item $Z(b) = b(42 - b^{2})$ \label{zeromultgraphlast}

\setcounter{HW}{\value{enumi}}
\end{enumerate}
%\end{multicols}

%\end{document}



In Exercises \ref{polytransfirst} - \ref{polytranslast}, given the pair of functions $f$ and $g$, sketch the graph of $y=g(x)$ by starting with the graph of $y = f(x)$ and using transformations.  Track at least three points of your choice through the transformations. State the domain and range of $g$.

%\begin{multicols}{2}
\begin{enumerate}
\setcounter{enumi}{\value{HW}}

\item $f(x) = x^3$,  $g(x) = (x + 2)^{3} + 1$ \label{polytransfirst}
\item $f(x) = x^4$, $g(x) = (x + 2)^{4} + 1$

\setcounter{HW}{\value{enumi}}
\end{enumerate}
%\end{multicols}

%\begin{multicols}{2}
\begin{enumerate}
\setcounter{enumi}{\value{HW}}

\item $f(x) = x^4$, $g(x) = 2 - 3(x - 1)^{4}$
\item $f(x) = x^5$, $g(x) = -x^{5} - 3$

\setcounter{HW}{\value{enumi}}
\end{enumerate}
%\end{multicols}

%\begin{multicols}{2}
\begin{enumerate}
\setcounter{enumi}{\value{HW}}

\item $f(x) = x^5$, $g(x) = (x+1)^5+10$
\item $f(x) = x^6$, $g(x) = 8-x^6$ \label{polytranslast}

\setcounter{HW}{\value{enumi}}
\end{enumerate}
%\end{multicols}

%\end{document}


\begin{enumerate}
\setcounter{enumi}{\value{HW}}

\item Use the Intermediate Value Theorem to prove that $f(x) = x^{3} - 9x + 5$ has a real zero in each of the following intervals: $[-4, -3], [0, 1]$ and $[2, 3]$.

\item  Rework Example \ref{boxnotopex} assuming the box is to be made from an 8.5 inch by 11 inch sheet of paper. Using scissors and tape, construct the box.  Are you surprised?\footnote{Consider decorating the box and presenting it to your instructor. If done well enough, maybe your instructor will issue you some bonus points.  Or maybe not.}

\setcounter{HW}{\value{enumi}}
\end{enumerate}


\phantomsection
\label{LCDmaxprofit} 

In Exercises \ref{lcdmaxprofitexerfirst} - \ref{lcdmaxprofitexerlast}, suppose the revenue $R$, in \textit{thousands} of dollars, from producing and selling $x$ \textit{hundred} LCD TVs is given by $R(x) = -5x^3+35x^2+155x$ for $0 \leq x \leq 10.07$.

\begin{enumerate}
\setcounter{enumi}{\value{HW}}

\item  Use a graphing utility to graph $y = R(x)$ and determine the number of TVs which should be sold to maximize revenue.  What is the maximum revenue? \label{lcdmaxprofitexerfirst}

\item Assume that the cost, in \textit{thousands} of dollars, to produce $x$ \textit{hundred} LCD TVs is given by $C(x) = 200x + 25$ for $x \geq 0$. Find and simplify an expression for the profit function $P(x)$.  (Remember: Profit = Revenue - Cost.)

\item  Use a graphing utility to graph $y = P(x)$ and determine the number of TVs which should be sold to maximize profit.  What is the maximum  profit? \label{lcdmaxprofitexerlast}

\item \label{newportaboycost} While developing their newest game, Sasquatch Attack!, the makers of the PortaBoy (from Example \ref{PortaBoyCost}) revised their cost function and now use $C(x) = .03x^{3} - 4.5x^{2} + 225x + 250$, for $x \geq 0$. As before, $C(x)$ is the cost to make $x$ PortaBoy Game Systems.  Market research indicates that the demand function $p(x) = -1.5x + 250$ remains unchanged.  Use a graphing utility to find the production level $x$ that maximizes the \textit{profit} made by producing and selling $x$ PortaBoy game systems.

\setcounter{HW}{\value{enumi}}
\end{enumerate}

\begin{enumerate}
\setcounter{enumi}{\value{HW}}

\item According to US Postal regulations, a rectangular shipping box must satisfy the inequality ``Length + Girth $\leq$ 130 inches'' for Parcel Post and ``Length + Girth $\leq$ 108 inches'' for other services.\footnote{See \href{http://www.usps.com/send/preparemailandpackages/measuringtips.htm}{\underline{here}} for details.}  Let's assume we have a closed rectangular box with a square face of side length $x$ as drawn below.  The length is the longest side and is clearly labeled.  The girth is the distance around the box in the other two dimensions so in our case it is the sum of the four sides of the square, $4x$.  

\begin{enumerate}

\item \label{girthbox1} Assuming that we'll be mailing a box via Parcel Post where Length + Girth $=$ 130 inches, express the length of the box in terms of $x$ and then express the volume $V$ of the box in terms of $x$.

\item \label{girthbox2} Find the dimensions of the box of maximum volume that can be shipped via Parcel Post.

\item Repeat parts \ref{girthbox1} and \ref{girthbox2} if the box is shipped using ``other services''.

\end{enumerate}

%\begin{center}
%
%\begin{mfpic}[8]{-6}{12}{-1}{17}
%\polyline{(0,0),(-4,3)}
%\polyline{(-4,3), (-4,8)}
%\polyline{(-4,8),(0,5)}
%\polyline{(0,5),(0,0)}
%\polyline{(0,0),(12,9)}
%\polyline{(0,5),(12,14)}
%\polyline{(-4,8),(8,17)}
%\polyline{(8,17),(12,14)}
%\polyline{(12,14),(12,9)}
%\arrow \reverse \arrow \polyline{(0,-0.5),(12,8.5)}
%\tlabel[cc](8,4){\tiny length}
%\arrow \reverse \arrow \polyline{(-4, 2.5), (-0.5,-0.125)}
%\tlabel[cc](-3,0.5){\tiny $x$}
%\arrow \reverse \arrow \polyline{(-4.5, 3), (-4.5,8)}
%\tlabel[cc](-5,5){\tiny $x$}
%\end{mfpic}
%
%\end{center}

\setcounter{HW}{\value{enumi}}
\end{enumerate}

\begin{enumerate}
\setcounter{enumi}{\value{HW}}


\item We now revisit the data set from Exercise \ref{regsunlight} in Section \ref{Regression}.  In that exercise, you were given a chart of the number of hours of daylight they get on the $21^{\mbox{st}}$ of each month in Fairbanks, Alaska based on the 2009 sunrise and sunset data found on the  \href{http://aa.usno.navy.mil/data/docs/RS_OneYear.php}{\underline{U.S. Naval Observatory}} website.  We let $x = 1$ represent January 21, 2009, $x = 2$ represent February 21, 2009, and so on.  The chart is given again for reference.

\medskip

\small

\noindent \begin{tabular}{|l|r|r|r|r|r|r|r|r|r|r|r|r|} \hline
Month  & & & & & & & & & & & & \\
Number & 1 & 2 & 3 & 4 & 5 & 6 & 7 & 8 & 9 & 10 & 11 & 12\\ 
\hline 
Hours of  & & & & & & & & & & & & \\
Daylight & 5.8 & 9.3 & 12.4 & 15.9 & 19.4 & 21.8 & 19.4 & 15.6 & 12.4 & 9.1 & 5.6 & 3.3 \\ \hline
\end{tabular}

\normalsize

\medskip

\noindent Find cubic (third degree) and quartic (fourth degree) polynomials which model this data and comment on the goodness of fit for each.  What can we say about using either model to make predictions about the year 2020?  (Hint: Think about the end behavior of polynomials.)  Use the models to see how many hours of daylight they got on your birthday and then check the website to see how accurate the models are.  Knowing that Sasquatch are largely nocturnal, what days of the year according to your models are going to allow for at least 14 hours of darkness for field research on the elusive creatures? 

\item \label{circuitexercisepoly} An electric circuit is built with a variable resistor installed.  For each of the following resistance values (measured in kilo-ohms, $k \Omega$),  the corresponding power to the load (measured in milliwatts, $mW$) is given in the table below. \footnote{The authors wish to thank Don Anthan and Ken White of Lakeland Community College for devising this problem and generating the accompanying data set.}

\noindent \begin{tabular}{|l|r|r|r|r|r|r|} \hline
Resistance: ($k \Omega$) & 1.012 & 2.199 & 3.275 & 4.676 & 6.805 & 9.975 \\ \hline
Power: ($mW$) & 1.063 & 1.496 & 1.610 & 1.613 & 1.505 & 1.314 \\ \hline
\end{tabular}

\begin{enumerate}

\item Make a scatter diagram of the data using the Resistance as the independent variable and Power as the dependent variable.

\item Use your calculator to find quadratic (2nd degree), cubic (3rd degree) and quartic (4th degree) regression models for the data and judge the reasonableness of each.

\item For each of the models found above, find the predicted maximum power that can be delivered to the load.  What is the corresponding resistance value?

\item Discuss with your classmates the limitations of these models - in particular, discuss the end behavior of each.

\end{enumerate}

\item Show that the end behavior of a linear function $f(x) = mx + b$ is as it should be according to the results we've established in the section for polynomials of odd degree.\footnote{Remember, to be a linear function, $m \neq 0$.}  (That is, show that the graph of a linear function is ``up on one side and down on the other'' just like the graph of $y = a_{n}x^{n}$ for odd numbers $n$.)

\item \index{multiplicity ! effect on the graph of a polynomial} There is one subtlety about the role of multiplicity that we need to discuss further; specifically we need to see `how' the graph crosses the $x$-axis at a zero of odd multiplicity.  In the section, we deliberately excluded the function $f(x) = x$ from the discussion of the end behavior of $f(x) = x^{n}$ for odd numbers $n$ and we said at the time that it was due to the fact that $f(x) = x$ didn't fit the pattern we were trying to establish.  You just showed in the previous exercise that the end behavior of a linear function behaves like every other polynomial of odd degree, so what doesn't $f(x) = x$ do that $g(x) = x^{3}$ does?  It's the `flattening' for values of $x$ near zero.  It is this local behavior that will distinguish between a zero of multiplicity 1 and one of higher odd multiplicity.  Look again closely at the graphs of $a(x) = x(x + 2)^{2}$ and $F(x) = x^{3}(x + 2)^{2}$ from Exercise \ref{polygraphexercise}.  Discuss with your classmates how the graphs are fundamentally different at the origin.  It might help to use a graphing calculator to zoom in on the origin to see the different crossing behavior. Also compare the behavior of $a(x) = x(x + 2)^{2}$ to that of $g(x) = x(x + 2)^{3}$ near the point $(-2, 0)$.  What do you predict will happen at the zeros of $f(x) = (x - 1)(x - 2)^2(x - 3)^{3}(x - 4)^{4}(x - 5)^{5}$?

\item Here are a few other questions for you to discuss with your classmates.  

\begin{enumerate}

\item How many local extrema could a polynomial of degree $n$ have?  How few local extrema can it have?
\item Could a polynomial have two local maxima but no local minima?  
\item If a polynomial has two local maxima and two local minima, can it be of odd degree?  Can it be of even degree?
\item Can a polynomial have local extrema without having any real zeros?
\item Why must every polynomial of odd degree have at least one real zero?
\item Can a polynomial have two distinct real zeros and no local extrema?
\item Can an $x$-intercept yield a local extrema?  Can it yield an absolute extrema?
\item If the $y$-intercept yields an absolute minimum, what can we say about the degree of the polynomial and the sign of the leading coefficient?   

\end{enumerate}

\setcounter{HW}{\value{enumi}}
\end{enumerate}
\newpage

\subsection{Answers}

%\begin{multicols}{2}
\begin{enumerate}

\item $f(x) = 4-x-3x^2$ \\
Degree 2 \\
Leading term $-3x^{2}$\\
Leading coefficient $-3$\\
Constant term $4$\\
As $x \rightarrow -\infty, \; f(x) \rightarrow -\infty$\\
As $x \rightarrow \infty, \; f(x) \rightarrow -\infty$\\

\item  $g(x) = 3x^5 - 2x^2 + x + 1$ \\
Degree 5 \\
Leading term $3x^5$\\
Leading coefficient $3$\\
Constant term $1$\\
As $x \rightarrow -\infty, \; g(x) \rightarrow -\infty$\\
As $x \rightarrow \infty, \; g(x) \rightarrow \infty$\\


\setcounter{HW}{\value{enumi}}
\end{enumerate}
%\end{multicols}

%\begin{multicols}{2}
\begin{enumerate}
\setcounter{enumi}{\value{HW}}

\item $q(r) = 1 - 16r^{4}$\\
Degree 4 \\
Leading term $-16r^{4}$\\
Leading coefficient $-16$\\
Constant term $1$\\
As $r \rightarrow -\infty, \; q(r) \rightarrow -\infty$\\
As $r \rightarrow \infty, \; q(r) \rightarrow -\infty$\\

\item $Z(b) = 42b - b^{3}$\\
Degree 3 \\
Leading term $-b^{3}$\\
Leading coefficient $-1$\\
Constant term $0$\\
As $b \rightarrow -\infty, \; Z(b) \rightarrow \infty$\\
As $b \rightarrow \infty, \; Z(b) \rightarrow -\infty$\\

\setcounter{HW}{\value{enumi}}
\end{enumerate}
%\end{multicols}

%\begin{multicols}{2}
\begin{enumerate}
\setcounter{enumi}{\value{HW}}

\item $f(x) = \sqrt{3}x^{17} + 22.5x^{10} - \pi x^{7} + \frac{1}{3}$\\
Degree 17 \\
Leading term $\sqrt{3}x^{17}$\\
Leading coefficient $\sqrt{3}$\\
Constant term $\frac{1}{3}$\\
As $x \rightarrow -\infty, \; f(x) \rightarrow -\infty$\\
As $x \rightarrow \infty, \; f(x) \rightarrow \infty$\\


\item $s(t) = -4.9t^{2} + v_{\mbox{\tiny $0$}}t + s_{\mbox{\tiny $0$}}$\\
Degree 2 \\
Leading term $-4.9t^{2}$\\
Leading coefficient $-4.9$\\
Constant term $s_{\mbox{\tiny $0$}}$\\
As $t \rightarrow -\infty, \; s(t) \rightarrow -\infty$\\
As $t \rightarrow \infty, \; s(t) \rightarrow -\infty$\\


\setcounter{HW}{\value{enumi}}
\end{enumerate}
%\end{multicols}

%\begin{multicols}{2}
\begin{enumerate}
\setcounter{enumi}{\value{HW}}


\item $P(x) = (x - 1)(x - 2)(x - 3)(x - 4)$\\
Degree 4 \\
Leading term $x^{4}$\\
Leading coefficient $1$\\
Constant term $24$\\
As $x \rightarrow -\infty, \; P(x) \rightarrow \infty$\\
As $x \rightarrow \infty, \; P(x) \rightarrow \infty$\\

\item $p(t) = -t^2(3 - 5t)(t^{2} + t + 4)$\\
Degree 5 \\
Leading term $5t^{5}$\\
Leading coefficient $5$\\
Constant term $0$\\
As $t \rightarrow -\infty, \; p(t) \rightarrow -\infty$\\
As $t \rightarrow \infty, \; p(t) \rightarrow \infty$\\

\setcounter{HW}{\value{enumi}}
\end{enumerate}
%\end{multicols}

\pagebreak

%\begin{multicols}{2}
\begin{enumerate}
\setcounter{enumi}{\value{HW}}

\item $f(x) = -2x^3(x+1)(x+2)^2$ \\
Degree 6 \\
Leading term $-2x^{6}$\\
Leading coefficient $-2$\\
Constant term $0$\\
As $x \rightarrow -\infty, \; f(x) \rightarrow -\infty$\\
As $x \rightarrow \infty, \; f(x) \rightarrow -\infty$\\

\item $G(t) = 4(t-2)^2\left(t+\frac{1}{2}\right)$ \\
Degree 3 \\
Leading term $4t^3$\\
Leading coefficient $4$\\
Constant term $8$\\
As $t \rightarrow -\infty, \; G(t) \rightarrow -\infty$\\
As $t \rightarrow \infty, \; G(t) \rightarrow \infty$\\

\setcounter{HW}{\value{enumi}}
\end{enumerate}
%\end{multicols}

%\begin{multicols}{2}
\begin{enumerate}
\setcounter{enumi}{\value{HW}}

\item $a(x) = x(x + 2)^{2}$\\
$x = 0$ multiplicity 1\\
$x = -2$ multiplicity 2\\

%\begin{mfpic}[20][10]{-3}{1}{-3}{5}
%\arrow \reverse \arrow \function{-3,0.65,0.1}{x*((x + 2)**2)}
%\axes
%\tlabel[cc](1,-0.5){\scriptsize $x$}
%\tlabel[cc](0.25,5){\scriptsize $y$}
%\point[3pt]{(-2,0), (0,0)}
%\xmarks{-2,-1}
%\tiny
%\tlpointsep{4pt}
%\axislabels {x}{{$-2 \hspace{6pt}$} -2, {$-1 \hspace{6pt}$} -1}
%\normalsize
%\end{mfpic}

\vfill

%\columnbreak

\item $g(x) = x(x + 2)^{3}$\\
$x = 0$ multiplicity 1\\
$x = -2$ multiplicity 3\\

%\begin{mfpic}[20][20]{-3}{1}{-2}{5}
%\arrow \reverse \arrow \function{-3,0.3,0.1}{x*((x + 2)**3)}
%\axes
%\tlabel[cc](1,-0.5){\scriptsize $x$}
%\tlabel[cc](0.25,5){\scriptsize $y$}
%\point[3pt]{(-2,0), (0,0)}
%\xmarks{-2,-1}
%\tiny
%\tlpointsep{4pt}
%\axislabels {x}{{$-2 \hspace{6pt}$} -2, {$-1 \hspace{6pt}$} -1}
%\normalsize
%\end{mfpic}


\setcounter{HW}{\value{enumi}}
\end{enumerate}
%\end{multicols}

%\begin{multicols}{2}
\begin{enumerate}
\setcounter{enumi}{\value{HW}}

\item $f(x) = -2(x-2)^2(x+1)$\\
$x=2$ multiplicity 2 \\
$x=-1$ multiplicity 1\\

%\begin{mfpic}[20][10]{-3}{3}{-4}{4}
%\arrow \reverse \arrow \function{-1.70,3.45,0.1}{(-0.4)*((x-2)**2)*(x+1)}
%\axes
%\tlabel[cc](3,-0.5){\scriptsize $x$}
%\tlabel[cc](0.25,4){\scriptsize $y$}
%\point[3pt]{(2,0), (-1,0)}
%\xmarks{-2,-1,1,2}
%\tiny
%\tlpointsep{4pt}
%\axislabels {x}{{$-2 \hspace{6pt}$} -2, {$-1 \hspace{6pt}$} -1, {$1$} 1, {$2$} 2}
%\normalsize
%\end{mfpic}



\item $g(x) = (2x+1)^2(x-3)$\\
$x=-\frac{1}{2}$ multiplicity 2 \\
$x=3$ multiplicity 1\\

%\begin{mfpic}[20][10]{-2}{4}{-4}{4}
%\arrow \reverse \arrow \function{-1.5,3.3,0.1}{(0.5)*((x+0.5)**2)*(x-3)}
%\axes
%\tlabel[cc](4,-0.5){\scriptsize $x$}
%\tlabel[cc](0.25,4){\scriptsize $y$}
%\point[3pt]{(-0.5,0), (3,0)}
%\xmarks{-1,1,2,3}
%\tiny
%\tlpointsep{4pt}
%\axislabels {x}{{$-1 \hspace{6pt}$} -1, {$1$} 1, {$2$} 2, {$3$} 3}
%\normalsize
%\end{mfpic}



\setcounter{HW}{\value{enumi}}
\end{enumerate}
%\end{multicols}

\pagebreak

%\begin{multicols}{2}
\begin{enumerate}
\setcounter{enumi}{\value{HW}}

\item $F(x) = x^{3}(x + 2)^{2}$\\
$x = 0$ multiplicity 3\\
$x = -2$ multiplicity 2\\

%\begin{mfpic}[20][10]{-3}{1}{-3}{5}
%\arrow \reverse \arrow \function{-2.45,0.85,0.1}{(x**3)*((x + 2)**2)}
%\axes
%\tlabel[cc](1,-0.5){\scriptsize $x$}
%\tlabel[cc](0.25,5){\scriptsize $y$}
%\point[3pt]{(-2,0), (0,0)}
%\xmarks{-2,-1}
%\tiny
%\tlpointsep{4pt}
%\axislabels {x}{{$-2 \hspace{6pt}$} -2, {$-1 \hspace{6pt}$} -1}
%\normalsize
%\end{mfpic}

\vfill

%\columnbreak

\item $P(x) = (x - 1)(x - 2)(x - 3)(x - 4)$\\
$x = 1$ multiplicity 1\\
$x = 2$ multiplicity 1\\
$x = 3$ multiplicity 1\\
$x = 4$ multiplicity 1\\

%\begin{mfpic}[20][10]{0}{5}{-1}{5}
%\arrow \reverse \arrow \function{0.6,4.4,0.1}{(x - 1)*(x - 2)*(x - 3)*(x - 4)}
%\axes
%\tlabel[cc](5,-0.5){\scriptsize $x$}
%\tlabel[cc](0.25,5){\scriptsize $y$}
%\point[3pt]{(1,0),(2,0),(3,0),(4,0)}
%\xmarks{1,2,3,4}
%\tiny
%\tlpointsep{4pt}
%\axislabels {x}{{$1$} 1, {$2$} 2, {$3$} 3, {$4$} 4}
%\normalsize
%\end{mfpic}

\setcounter{HW}{\value{enumi}}
\end{enumerate}
%\end{multicols}


%\begin{multicols}{2}
\begin{enumerate}
\setcounter{enumi}{\value{HW}}


\item $Q(x) = (x + 5)^{2}(x - 3)^{4}$\\
$x = -5$ multiplicity 2\\
$x = 3$ multiplicity 4\\

%\begin{mfpic}[10][20]{-6}{6}{-1}{3}
%\arrow \reverse \arrow \function{-5.9,5.6,0.1}{(((x + 5)**2)*((x - 3)**4))/2000}
%\axes
%\tlabel[cc](6,-0.5){\scriptsize $x$}
%\tlabel[cc](0.5,3){\scriptsize $y$}
%\point[3pt]{(-5,0),(3,0)}
%\xmarks{-5 step 1 until 5}
%\tiny
%\tlpointsep{4pt}
%\axislabels {x}{{$-5 \hspace{6pt}$} -5, {$-4 \hspace{6pt}$} -4, {$-3 \hspace{6pt}$} -3, {$-2 \hspace{6pt}$} -2, {$-1 \hspace{6pt}$} -1, {$1$} 1, {$2$} 2, {$3$} 3, {$4$} 4, {$5$} 5}
%\normalsize
%\end{mfpic}

\vfill

%\columnbreak

\item $f(x) = x^2(x-2)^2(x+2)^2$\\
$x = -2$ multiplicity 2\\
$x = 0$ multiplicity 2\\
$x = 2$ multiplicity 2\\

%\begin{mfpic}[20][10]{-3}{3}{-1}{5}
%\arrow \reverse \arrow \function{-2.45,2.45,0.1}{(0.2)*(x**2)*((x-2)**2)*((x+2)**2)}
%\axes
%\tlabel[cc](3,-0.5){\scriptsize $x$}
%\tlabel[cc](0.5,5){\scriptsize $y$}
%\point[3pt]{(-2,0), (0,0), (2,0)}
%\xmarks{-2 step 1 until 2}
%\tiny
%\tlpointsep{4pt}
%\axislabels {x}{{$-2 \hspace{6pt}$} -2, {$-1 \hspace{6pt}$} -1, {$1$} 1, {$2$} 2}
%\normalsize
%\end{mfpic}

\setcounter{HW}{\value{enumi}}
\end{enumerate}
%\end{multicols}

%\begin{multicols}{2}
\begin{enumerate}
\setcounter{enumi}{\value{HW}}

\item $H(t) = (3-t)\left(t^2+1\right)$\\
$x =3$ multiplicity 1\\

%\begin{mfpic}[20][10]{-1}{4}{-4}{4}
%\arrow \reverse \arrow \function{-0.75,3.3,0.1}{(0.5)*(3-x)*((x**2)+1)}
%\axes
%\tlabel[cc](4,-0.5){\scriptsize $t$}
%\tlabel[cc](0.5,3){\scriptsize $y$}
%\point[3pt]{(3,0)}
%\xmarks{1 step 1 until 3}
%\tiny
%\tlpointsep{4pt}
%\axislabels {x}{{$1$} 1, {$2$} 2, {$3$} 3}
%\normalsize
%\end{mfpic}

\vfill

%\columnbreak

\item $Z(b) = b(42 - b^{2})$\\
$b = -\sqrt{42}$ multiplicity 1\\
$b = 0$ multiplicity 1\\
$b = \sqrt{42}$ multiplicity 1\\

%\begin{mfpic}[10]{-7}{7}{-6}{6}
%\arrow \reverse \arrow \function{-7,7,0.1}{(42*x - x**3)/20}
%\axes
%\tlabel[cc](7,-0.5){\scriptsize $b$}
%\tlabel[cc](0.5,6){\scriptsize $y$}
%\point[3pt]{(-6.4807,0),(0,0),(6.4807,0)}
%\xmarks{-6 step 1 until 6}
%\tiny
%\tlpointsep{4pt}
%\axislabels {x}{{$-6 \hspace{6pt}$} -6, {$-5 \hspace{6pt}$} -5, {$-4 \hspace{6pt}$} -4, {$-3 \hspace{6pt}$} -3, {$-2 \hspace{6pt}$} -2, {$-1 \hspace{6pt}$} -1, {$1$} 1, {$2$} 2, {$3$} 3, {$4$} 4, {$5$} 5, {$6$} 6}
%\normalsize
%\end{mfpic}

\setcounter{HW}{\value{enumi}}
\end{enumerate}
%\end{multicols}

\pagebreak

%\begin{multicols}{2}
\begin{enumerate}
\setcounter{enumi}{\value{HW}}

\item $g(x) = (x + 2)^{3} + 1$ \\ 
domain: $(-\infty, \infty)$ \\ 
range: $(-\infty, \infty)$ \\

%\begin{mfpic}[20][8]{-5}{1}{-11}{13}
%\arrow \reverse \arrow \function{-4.25,0.25,0.1}{((x + 2)**3) + 1}
%\axes
%\tlabel[cc](1,-0.75){\scriptsize $x$}
%\tlabel[cc](0.5,13){\scriptsize $y$}
%\point[3pt]{(-4,-7),(-3,0),(-2,1),(-1,2),(0,9)}
%\xmarks{-4,-3,-2,-1}
%\ymarks{-10 step 1 until 12}
%\tiny
%\tlpointsep{4pt}
%\axislabels {x}{{$-4 \hspace{6pt}$} -4, {$-3 \hspace{6pt}$} -3, {$-2 \hspace{6pt}$} -2, {$-1 \hspace{6pt}$} -1}
%\axislabels {y}{{$-10$} -10, {$-9$} -9, {$-8$} -8, {$-7$} -7, {$-6$} -6, {$-5$} -5, {$-4$} -4, {$-3$} -3, {$-2$} -2, {$-1$} -1, {$1$} 1, {$2$} 2, {$3$} 3, {$4$} 4, {$5$} 5, {$6$} 6, {$7$} 7, {$8$} 8, {$9$} 9, {$10$} 10, {$11$} 11, {$12$} 12}
%\normalsize
%\end{mfpic}

\vfill

%\columnbreak

\item $g(x) = (x + 2)^{4} + 1$\\
domain: $(-\infty, \infty)$\\
range: $[1, \infty)$\\

%\begin{mfpic}[20][8]{-5}{1}{-1}{22}
%\arrow \reverse \arrow \function{-4.12,0.12,0.1}{((x + 2)**4) + 1}
%\axes
%\tlabel[cc](1,-0.75){\scriptsize $x$}
%\tlabel[cc](0.5,22){\scriptsize $y$}
%\point[3pt]{(-4,17),(-3,2),(-2,1),(-1,2),(0,17)}
%\xmarks{-4,-3,-2,-1}
%\ymarks{1 step 1 until 21}
%\tiny
%\tlpointsep{4pt}
%\axislabels {x}{{$-4 \hspace{6pt}$} -4, {$-3 \hspace{6pt}$} -3, {$-2 \hspace{6pt}$} -2, {$-1 \hspace{6pt}$} -1}
%\axislabels {y}{{$1$} 1, {$2$} 2, {$3$} 3, {$4$} 4, {$5$} 5, {$6$} 6, {$7$} 7, {$8$} 8, {$9$} 9, {$10$} 10, {$11$} 11, {$12$} 12, {$13$} 13, {$14$} 14, {$15$} 15, {$16$} 16, {$17$} 17, {$18$} 18, {$19$} 19, {$20$} 20, {$21$} 21}
%\normalsize
%\end{mfpic}

\setcounter{HW}{\value{enumi}}
\end{enumerate}
%\end{multicols}

%\begin{multicols}{2}
\begin{enumerate}
\setcounter{enumi}{\value{HW}}


\item $g(x) = 2 - 3(x - 1)^{4}$\\
domain: $(-\infty, \infty)$\\
range: $(-\infty, 2]$\\

%\begin{mfpic}[20][8]{-1}{3}{-14}{3}
%\arrow \reverse \arrow \function{-0.5,2.5,0.1}{2 - 3*((x - 1)**4)}
%\axes
%\tlabel[cc](3,-0.75){\scriptsize $x$}
%\tlabel[cc](0.5,3){\scriptsize $y$}
%\point[3pt]{(1,2),(0,-1),(2,-1)}
%\xmarks{1,2}
%\ymarks{-13 step 1 until 2}
%\tiny
%\tlpointsep{4pt}
%\axislabels {x}{{$1$} 1, {$2$} 2}
%\axislabels {y}{{$-13$} -13, {$-12$} -12, {$-11$} -11, {$-10$} -10, {$-9$} -9, {$-8$} -8, {$-7$} -7, {$-6$} -6, {$-5$} -5, {$-4$} -4, {$-3$} -3, {$-2$} -2, {$-1$} -1, {$1$} 1, {$2$} 2}
%\normalsize
%\end{mfpic}


\vfill

%\columnbreak

\item $g(x) = -x^{5} - 3$\\
domain: $(-\infty, \infty)$\\
range: $(-\infty, \infty)$\\

%\begin{mfpic}[20][8]{-2}{2}{-11}{11}
%\arrow \reverse \arrow \function{-1.68,1.5,0.1}{-(x**5) - 3}
%\axes
%\tlabel[cc](2,-0.75){\scriptsize $x$}
%\tlabel[cc](0.5,11){\scriptsize $y$}
%\point[3pt]{(-1,-2),(0,-3),(1,-4)}
%\xmarks{-1,1}
%\ymarks{-10 step 1 until 10}
%\tiny
%\tlpointsep{4pt}
%\axislabels {x}{{$-1 \hspace{6pt}$} -1, {$1$} 1}
%\axislabels {y}{{$-10$} -10, {$-9$} -9, {$-8$} -8, {$-7$} -7, {$-6$} -6, {$-5$} -5, {$-4$} -4, {$-3$} -3, {$-2$} -2, {$-1$} -1, {$1$} 1, {$2$} 2, {$3$} 3, {$4$} 4, {$5$} 5, {$6$} 6, {$7$} 7, {$8$} 8, {$9$} 9, {$10$} 10}
%\normalsize
%\end{mfpic}


\setcounter{HW}{\value{enumi}}
\end{enumerate}
%\end{multicols}


\pagebreak

%\begin{multicols}{2}
\begin{enumerate}

\setcounter{enumi}{\value{HW}}
\item $g(x) = (x+1)^5+10$\\
domain: $(-\infty, \infty)$\\
range: $(-\infty, \infty)$\\

%\begin{mfpic}[20][8]{-5}{1}{-1}{22}
%\arrow \reverse \arrow \function{-2.64,0.58,0.1}{((x + 1)**5) + 10}
%\axes
%\tlabel[cc](1,-0.75){\scriptsize $x$}
%\tlabel[cc](0.5,22){\scriptsize $y$}
%\point[3pt]{(0,11), (-1,10), (-2,9)}
%\xmarks{-4,-3,-2,-1}
%\ymarks{1 step 1 until 21}
%\tiny
%\tlpointsep{4pt}
%\axislabels {x}{{$-4 \hspace{6pt}$} -4, {$-3 \hspace{6pt}$} -3, {$-2 \hspace{6pt}$} -2, {$-1 \hspace{6pt}$} -1}
%\axislabels {y}{{$1$} 1, {$2$} 2, {$3$} 3, {$4$} 4, {$5$} 5, {$6$} 6, {$7$} 7, {$8$} 8, {$9$} 9, {$10$} 10, {$11$} 11, {$12$} 12, {$13$} 13, {$14$} 14, {$15$} 15, {$16$} 16, {$17$} 17, {$18$} 18, {$19$} 19, {$20$} 20, {$21$} 21}
%\normalsize
%\end{mfpic}
%

\vfill

%\columnbreak

\item $g(x) = 8-x^{6}$\\
domain: $(-\infty, \infty)$\\
range: $(-\infty, 8]$\\

%\begin{mfpic}[20][8]{-2}{2}{-11}{11}
%\arrow \reverse \arrow \function{-1.6,1.6,0.1}{8-(x**6)}
%\axes
%\tlabel[cc](2,-0.75){\scriptsize $x$}
%\tlabel[cc](0.5,11){\scriptsize $y$}
%\point[3pt]{(-1,7),(0,8),(1,7)}
%\xmarks{-1,1}
%\ymarks{-10 step 1 until 10}
%\tiny
%\tlpointsep{4pt}
%\axislabels {x}{{$-1 \hspace{6pt}$} -1, {$1$} 1}
%\axislabels {y}{{$-10$} -10, {$-9$} -9, {$-8$} -8, {$-7$} -7, {$-6$} -6, {$-5$} -5, {$-4$} -4, {$-3$} -3, {$-2$} -2, {$-1$} -1, {$1$} 1, {$2$} 2, {$3$} 3, {$4$} 4, {$5$} 5, {$6$} 6, {$7$} 7, {$8$} 8, {$9$} 9, {$10$} 10}
%\normalsize
%\end{mfpic}


\setcounter{HW}{\value{enumi}}
\end{enumerate}
%\end{multicols}

\begin{enumerate}
\setcounter{enumi}{\value{HW}}

\item We have $f(-4)=-23,\; f(-3)=5,\; f(0)=5,\; f(1)=-3,\; f(2)=-5\;$ and $f(3)=5$ so the Intermediate Value Theorem tells us that $f(x) = x^{3} - 9x + 5$ has real zeros in the intervals $[-4, -3], [0, 1]$ and $[2, 3]$.

\item  $V(x) = x(8.5-2x)(11-2x) = 4x^3-39x^2+93.5x$, $0 < x < 4.25$.  Volume is maximized when $x \approx 1.58$, so the dimensions of the box with maximum volume are: height $\approx$ 1.58 inches, width $\approx$ 5.34 inches, and depth $\approx$ 7.84 inches.  The maximum volume is $\approx$ 66.15 cubic inches.


\item The calculator gives the location  of the absolute maximum (rounded to three decimal places) as $x \approx 6.305$ and $y \approx 1115.417$.  Since $x$ represents the number of TVs sold in hundreds, $x = 6.305$ corresponds to $630.5$ TVs.  Since we can't sell half of a TV, we compare $R(6.30) \approx 1115.415$ and $R(6.31) \approx 1115.416$, so selling $631$ TVs results in a (slightly) higher revenue.  Since $y$ represents the revenue in \textit{thousands} of dollars, the maximum revenue is $\$ 1,\!115,\!416$.

\item $P(x) = R(x) - C(x) = -5x^3+35x^2-45x-25$, $0 \leq x \leq 10.07$.

\item  The calculator gives the location  of the absolute maximum (rounded to three decimal places) as $x \approx 3.897$ and $y \approx 35.255$.  Since $x$ represents the number of TVs sold in hundreds, $x = 3.897$ corresponds to $389.7$ TVs.  Since we can't sell $0.7$ of a TV, we compare $P(3.89) \approx 35.254$ and $P(3.90) \approx 35.255$, so selling $390$ TVs results in a (slightly) higher revenue.  Since $y$ represents the revenue in \textit{thousands} of dollars, the maximum revenue is $\$ 35,\!255$.

\item Making and selling 71 PortaBoys yields a maximized profit of \$5910.67.


\item \begin{enumerate}

\item Our ultimate goal is to maximize the volume, so we'll start with the maximum Length $+$ Girth of $130.$  This means the length is $130 - 4x$.  The volume of a rectangular box is always length $\times$ width $\times$ height so we get $V(x) = x^{2}(130 - 4x) = -4x^{3} + 130x^{2}$.  

\item Graphing $y = V(x)$ on $[0, 33] \times [0, 21000]$ shows a maximum at $(21.67, 20342.59)$ so the dimensions of the box with maximum volume are $21.67\mbox{in.} \times 21.67\mbox{in.} \times 43.32\mbox{in.}$ for a volume of $20342.59\mbox{in.}^{3}$.

\item If we start with Length $+$ Girth $= 108$ then the length is $108 - 4x$ and the volume is $V(x) = -4x^{3} + 108x^{2}$.  Graphing $y = V(x)$ on $[0, 27] \times [0, 11700]$ shows a maximum at $(18.00, 11664.00)$ so the dimensions of the box with maximum volume are $18.00\mbox{in.} \times 18.00\mbox{in.} \times 36\mbox{in.}$ for a volume of $11664.00\mbox{in.}^{3}$.  (Calculus will confirm that the measurements which maximize the volume are \underline{exactly} 18in. by 18in. by 36in., however, as I'm sure you are aware by now, we treat all calculator results as approximations and list them as such.)

\end{enumerate}

\setcounter{HW}{\value{enumi}}
\end{enumerate}




\begin{enumerate}
\setcounter{enumi}{\value{HW}}


\item The cubic regression model is $p_{\mbox{\tiny $3$}}(x) = 0.0226x^{3} - 0.9508x^{2} + 8.615x - 3.446$.  It has $R^{2} = 0.93765$ which isn't bad.  The graph of $y = p_{\mbox{\tiny $3$}}(x)$ in the viewing window $[-1,13] \times [0, 24]$ along with the scatter plot is shown below on the left.  Notice that $p_{\mbox{\tiny $3$}}$ hits the $x$-axis at about $x = 12.45$ making this a bad model for future predictions.  To use the model to approximate the number of hours of sunlight on your birthday, you'll have to figure out what decimal value of $x$ is close enough to your birthday and then plug it into the model.  My (Jeff's) birthday is July 31 which is 10 days after July 21 ($x = 7$).  Assuming 30 days in a month, I think $x = 7.33$ should work for my birthday and $p_{\mbox{\tiny $3$}}(7.33) \approx 17.5$.  The website says there will be about $18.25$ hours of daylight that day.  To have 14 hours of darkness we need 10 hours of daylight.  We see that $p_{\mbox{\tiny $3$}}(1.96) \approx 10$ and $p_{\mbox{\tiny $3$}}(10.05) \approx 10$ so it seems reasonable to say that we'll have at least 14 hours of darkness from December 21, 2008 ($x = 0$) to February 21, 2009 ($x = 2$) and then again from October 21,2009 ($x = 10$) to December 21, 2009 ($x = 12$).

\smallskip

The quartic regression model is $p_{\mbox{\tiny $4$}}(x) = 0.0144x^{4} - 0.3507x^{3} + 2.259x^{2} - 1.571x + 5.513$.  It has $R^{2} = 0.98594$ which is good.  The graph of $y = p_{\mbox{\tiny $4$}}(x)$ in the viewing window $[-1, 15] \times [0, 35]$ along with the scatter plot is shown below on the right.  Notice that $p_{\mbox{\tiny $4$}}(15)$ is above $24$ making this a bad model as well for future predictions.  However, $p_{\mbox{\tiny $4$}}(7.33) \approx 18.71$ making it much better at predicting the hours of daylight on July 31 (my birthday).  This model says we'll have at least 14 hours of darkness from December 21, 2008 ($x = 0$) to about March 1, 2009 ($x = 2.30$) and then again from October 10, 2009 ($x = 9.667$) to December 21, 2009 ($x = 12$).

\begin{center}

\begin{tabular}{cc}

%\includegraphics[width=2in]{./PolynomialsGraphics/REG3.jpg} \hspace{.25in} & \includegraphics[width=2in]{./PolynomialsGraphics/REG4.jpg} \\

$y = p_{\mbox{\tiny $3$}}(x)$ \hspace{.25in} & $y = p_{\mbox{\tiny $4$}}(x)$ \\

\end{tabular}

\end{center}

\item \begin{enumerate}

\item The scatter plot is shown below with each of the three regression models.

\item The quadratic model is $P_{\mbox{\tiny $2$}}(x) = -0.02x^{2} + 0.241x + 0.956$ with $R^{2} = 0.77708$. \\
The cubic model is $P_{\mbox{\tiny $3$}}(x) = 0.005x^{3} - 0.103x^{2} + 0.602x + 0.573$ with $R^{2} = 0.98153$. \\
The quartic model is $P_{\mbox{\tiny $4$}}(x) = -0.000969x^{4} + 0.0253x^{3} - 0.240x^{2} + 0.944x + 0.330$ with $R^{2} = 0.99929$.

\item The maximums predicted by the three models are $P_{\mbox{\tiny $2$}}(5.737) \approx 1.648$, $P_{\mbox{\tiny $3$}}(4.232) \approx 1.657$ and $P_{\mbox{\tiny $4$}}(3.784) \approx 1.630$, respectively.

\end{enumerate}

\hspace{-.1in} \begin{tabular}{ccc}

%\includegraphics[width=1.8in]{./PolynomialsGraphics/CIRC2.jpg} \hspace{.1in} &
%\includegraphics[width=1.8in]{./PolynomialsGraphics/CIRC3.jpg} \hspace{.1in} &
%\includegraphics[width=1.8in]{./PolynomialsGraphics/CIRC4.jpg} \\

$y = P_{\mbox{\tiny $2$}}(x)$ \hspace{.1in} & $y = P_{\mbox{\tiny $3$}}(x)$ & $y = P_{\mbox{\tiny $4$}}(x)$\\

\end{tabular}

\end{enumerate}

\end{document}
